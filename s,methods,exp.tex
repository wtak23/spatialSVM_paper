%!TEX root = paper.tex
%===================================================================%
% Real data info
%===================================================================%
To further assess the utility of the proposed method, we also conducted experiments on real resting state scans.

\paragraph{Participants}
We used the Center for Biomedical Research Excellence (COBRE) dataset (\url{http://fcon_1000.projects.nitrc.org/indi/retro/cobre.html}) made available by the Mind Research Network. 
The dataset is comprised of $74$ typically developing control participants and $71$ participants with a DSM-IV-TR diagnosis of schizophrenia. 
Diagnosis was established by the Structured Clinical Interview for DSM-IV (SCID). Participants were excluded if they had mental retardation, neurological disorder, head trauma, or substance abuse or dependence in the last $12$ months. A summary of the participant demographic characteristics is provided in Table~\ref{table:schiz,demographic}. 

\paragraph{Data Acquisition}
A multi-echo MPRAGE (MEMPR) sequence was used with the following parameters: TR/TE/TI = $2530/[1.64, 3.5, 5.36, 7.22, 9.08]$/$900$ ms, flip angle $= 7^{\circ}$, FOV $= 256\times 256$ mm, slab thickness $= 176$ mm, matrix size $= 256\times 256\times 176$, voxel size $= 1\times 1\times 1$ mm, number of echoes $= 5$, pixel bandwidth $=650$ Hz, total scan time $= 6$ minutes. 
With $5$ echoes, the TR and TI time to encode partitions for the MEMPR are similar to that of a conventional MPRAGE, resulting in similar GM/WM/CSF contrast. 
Resting state data were collected with single-shot full k-space echo-planar imaging (EPI) with ramp sampling correction using the intercomissural line (AC-PC) as a reference (TR: $2$ s, TE: $29$ ms, matrix size: $64\times 64$, $32$ slices, voxel size: $3\times 3\times 4 mm^3$).

\paragraph{Imaging Sample Selection}
Analyses were limited to participants with: 
(1) MPRAGE anatomical images, with consistent near-full brain coverage (\ie, superior extent included the majority of frontal and parietal cortex and inferior extent included the temporal lobes) with successful registration; 
(2) complete phenotypic information for main phenotypic variables (diagnosis, age, handedness); 
(3) mean framewise displacement (FD) within two standard deviations of the sample mean; 
(4) at least $50\%$ of frames retained after application of framewise censoring for motion (``motion scrubbing''; see below).
After applying these sample selection criteria, we analyzed resting state scans from $121$ individuals consisting of $67$ healthy controls (HC) and $54$ schizophrenic subjects (SZ). 
Demographic characteristics of the post-exclusion sample are shown in Table~\ref{table:schiz,demographic}.

%=================================================================== 
\newcommand{\mycellcol}{\cellcolor{blue!25}}
\begin{table}[t!]
	\centering
	\begin{tabular}{l|cccc|cccc}
		\multicolumn{1}{c}{}&\multicolumn{4}{c}{\textbf{Healthy Controls}}&\multicolumn{4}{c}{\textbf{Schizophrenia}}\\
		\hline
			& $n$ & Age & \#male & \#RH & $n$ & Age & \#male & \#RH \\		
		\hline\hline
		 Pre-exclusion 	& $74$ & $35.8\pm 11.6$ & $51$ & $71$ 
		 		  		& $71$ & $38.1\pm 14.0$ & $57$ & $59$ \\
		 Post-exclusion 	& $67$ & $35.2\pm 11.7$ & $46$ & $66$ 
		 				& $54$ & $35.5\pm 13.1$ & $48$ & $46$ \\
		\hline
	\end{tabular}
	\caption{
		Demographic characteristics of the participants before and after sample exclusion criteria is applied \mbox{(RH = right-handed)}.
	}
	\label{table:schiz,demographic}
\end{table}
%=================================================================== 

\paragraph{Preprocessing}
Preprocessing steps were performed using statistical parametric mapping (SPM$8$; \url{www.fil.ion.ucl.ac.uk/spm}). 
Scans were reconstructed, slice-time corrected, realigned to the first scan in the experiment for correction of head motion, and co-registered with the high-resolution T$1$-weighted image. 
Normalization was performed using the voxel-based morphometry (VBM) toolbox implemented in SPM$8$. 
The high-resolution T$1$-weighted image was segmented into tissue types, bias-corrected, registered to MNI space, and then normalized using Diffeomorphic Anatomical Registration Through Exponentiated Lie Algebra (DARTEL) \citep{Ashburner:2007}. 
The resulting deformation fields were then applied to the functional images. 
Smoothing of functional data was performed with an $8$ mm$^3$ kernel.

\paragraph{Connectome generation}
Functional connectomes were generated by placing $7.5$ mm radius nodes representing ROIs encompassing $33$ $3\times 3\times 3$ mm voxels in a regular grid spaced at $18\times 18 \times 18$~mm intervals throughout the brain. 
Spatially averaged time series were extracted from each of the ROIs. 
Next, linear detrending was performed, followed by nuisance regression. 
Regressors included six motion regressors generated from the realignment step, as well as their first derivatives. 
White matter and cerebrospinal fluid masks were generated from the VBM-based tissue segmentation step noted above, and eroded using the \textsf{fslmaths} program from FSL to eliminate border regions of potentially ambiguous tissue type. 
The top five principal components of the BOLD time series were extracted from each of the masks and included as regressors in the model -- a method that has been demonstrated to effectively remove signals arising from the cardiac and respiratory cycle \citep{Behzadi:2007}. 
The time-series for each ROI was then band-passed filtered in the $0.01$ -- $0.10$ Hz range. 
Individual frames with excessive head motion were then censored from the time series. 
Subjects with more than $50\%$ of their frames removed by scrubbing were excluded from further analysis, a threshold justified by simulations conducted by other groups \citep{Fair:2013}, as well as by our group. 
Pearson product-moment correlation coefficients were then calculated pairwise between time courses for each of the $347$ ROIs.
Standard steps in functional connectivity analysis (removing motion artifacts and nuisance covariates and calculating Pearson's product moment correlations between pairs of nodes) was performed with \texttt{ConnTool}, a functional connectivity analysis package developed by Robert C. Welsh, University of Michigan.
